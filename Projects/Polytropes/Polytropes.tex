\documentclass{article}
\usepackage{amsmath}
\usepackage{graphicx}
\usepackage{listings}             % Include the listings-package
\usepackage{color}
\usepackage{xcolor}
%\usepackage[scaled]{beramono}
\usepackage{tgheros}
%\uspackage{mathtools}


\usepackage{caption}
\DeclareCaptionFont{white}{\color{white}}
\DeclareCaptionFormat{listing}{\colorbox{gray}{\parbox{\textwidth}{#1#2#3}}}
\captionsetup[lstlisting]{format=listing,labelfont=white,textfont=white}

\definecolor{mygreen}{rgb}{0,0.6,0}
\definecolor{mygray}{rgb}{0.5,0.5,0.5}
\definecolor{mymauve}{rgb}{0.58,0,0.82}
\definecolor{light-gray}{gray}{0.9}

\lstdefinestyle{FormattedNumber}{%
    literate={0}{{\textcolor{purple}{0}}}{1}%
             {1}{{\textcolor{purple}{1}}}{1}%
             {2}{{\textcolor{purple}{2}}}{1}%
             {3}{{\textcolor{purple}{3}}}{1}%
             {4}{{\textcolor{purple}{4}}}{1}%
             {5}{{\textcolor{purple}{5}}}{1}%
             {6}{{\textcolor{purple}{6}}}{1}%
             {7}{{\textcolor{purple}{7}}}{1}%
             {8}{{\textcolor{purple}{8}}}{1}%
             {9}{{\textcolor{purple}{9}}}{1}%
             {.0}{{\textcolor{purple}{.0}}}{2}% Following is to ensure that only periods
             {.1}{{\textcolor{purple}{.1}}}{2}% followed by a digit are changed.
             {.2}{{\textcolor{purple}{.2}}}{2}%
             {.3}{{\textcolor{purple}{.3}}}{2}%
             {.4}{{\textcolor{purple}{.4}}}{2}%
             {.5}{{\textcolor{purple}{.5}}}{2}%
             {.6}{{\textcolor{purple}{.6}}}{2}%
             {.7}{{\textcolor{purple}{.7}}}{2}%
             {.8}{{\textcolor{purple}{.8}}}{2}%
             {.9}{{\textcolor{purple}{.9}}}{2}%
     % ,
   %basicstyle=\ttfamily,%  Optional to use this
}



\lstset{    % Set your language (you can change the language for each code-block optionally)
    language=Python,
    basicstyle=\linespread{1.1}\ttfamily\footnotesize,
    backgroundcolor=\color{light-gray},
    commentstyle=\color{mygreen},
    keywordstyle=\color{blue},
    numbers=left,
    %numbersep=10pt,
    numberstyle=\tiny\color{mygray},
    rulecolor=\color{black},
    stringstyle=\color{mymauve},
    xleftmargin=0.7cm,
    frame=tlbr, framesep=0.2cm, framerule=0pt,
    showstringspaces=false
    }          

\begin{document}

\section{Introduction}
In this project you will write a computer program in python to solve a set of first order differential equations, used to model a spherical symmetric star in hydrostatic equilibrium. Throughout the text there are \textbf{problems} labeled in bold, make sure you do all of them. Most of the text serve as  a review, so if your comfortable with the material you can go straight to the problems. Each student is expected to hand in their own solutions.

\section{Physics}

\subsection{Pressure Integral}
Pressure is the force exerted on a given area. For gas, the pressure is due to the continual transfer of momentum by the gas particles. Consider a gas particle with momentum $p$ approaching the surface $dA$ at angle $\theta$ relative to the normal of the surface. The particle bounces off the surface elastically with momentum transfer of $2 p\mathrm{cos}\left(\theta\right)$ -- see Figure \ref{Fig: Momentum}.

\begin{figure}[h!]
\begin{center}
\includegraphics[scale=1.0]{Images/Momentum}
\end{center}
\caption{A particle traveling towards the surface $dA$. The particles has momentum $pcos\left(\theta\right)$ along the normal of
surface $dA$. The momentum transfer to the area is $\Delta p = 2pcos\left(\theta\right)$.}
\label{Fig: Momentum}
\end{figure}

Now lets imagine a beam of particles, what is the momentum transfer now? The momentum transfer is simply $2 p\mathrm{cos}(\theta)$ times the total number of particles that bounce off the surface in a time $dt$. So what is the rate of particles bouncing off the surface? Let $n$ be the number density of the particles, so then the number of particles is $n$ times the volume enclosing the total amount of particles that can reach the surface $dA$ in a time $dt$. The furthest particle that can reach $dA$ has a distance $v \mathrm{cos}(\theta) dt$, where $v$ is the velocity of the particles. Hence, the volume is $v\mathrm{cos}(\theta) dt\times dA$ -- see Figure \ref{Fig: Beam}.
\begin{figure}[h!]
\begin{center}
\includegraphics[scale=1.1]{Images/BeamParticles}
\end{center}
\caption{Beam of particles transferring momentum to the surface $dA$.}
\label{Fig: Beam}
\end{figure}
Therefore we have the total momentum transfer
\begin{equation*}
\begin{split}
dp & = (\mathrm{total\,\, particles})\times(\mathrm{momentum \,\, transfer}) \\
& = (n v\mathrm{cos}(\theta)dt dA)\times(2p \mathrm{cos}(\theta)),
\end{split}
\end{equation*}

\noindent
and the momentum transfer per area per time is
\begin{equation*}
\frac{d^2 p}{dA dt} = 2 n v p \mathrm{cos}^2(\theta).
\end{equation*}

At this point we have only accounted for a single beam but the pressure is due to the net affect of all the beams. To add the contribution of all beams let $dn/d\theta$ be the number density of particles coming at angles between $\theta$ and $\theta + d\theta$. If the distribution of particles is isotropic, meaning the same for any given $\theta$, then the fraction of particles at angle $\theta$ to $\theta +d\theta$ is proportional to the strip of area of an imaginary sphere of unit radius that lies between $\theta$ and $\theta + d\theta$ -- see Figure \ref{Fig: Solid Angle}. 

\begin{figure}[h!]
\begin{center}
\includegraphics[scale=1.1]{Images/SolidAngle}
\end{center}
\caption{Units sphere centered on $dA$. The fraction of particles at $\theta$ is proportional to the fraction of the solid angle that lies between $\theta$ and $\theta + d\theta$.}
\label{Fig: Solid Angle}
\end{figure}

\noindent
So the ratio of the area of the strip to the whole area must be equal to the number density fraction of particles coming at angles between $\theta$ and $\theta + d\theta$,

\begin{equation*}
\frac{dn/d\theta}{n} = \frac{\mathrm{area \,\, of \,\, strip}}{\mathrm{total \,\, area}} = \frac{2\pi \mathrm{\sin}(\theta) d\theta}{4\pi} =
\frac{1}{2} \mathrm{sin}(\theta) d\theta.
\end{equation*}

\noindent
So to take in account of all beams we have integrate over $\theta$

\begin{equation*}
\frac{d^2 p}{dA dt} = \int \frac{dn}{d\theta} 2 v p \mathrm{cos}^2(\theta) = \int_{0}^{\pi/2} \frac{1}{2} \mathrm{sin}(\theta) 2 n v p \mathrm{cos}^2 (\theta) d\theta.
\end{equation*}

\noindent
Lets do the integral, 
\begin{align*}
\int_{0}^{\pi/2} \mathrm{sin}(\theta) n v p \mathrm{cos}^2 (\theta) d\theta & = nvp \int_{0}^{\pi/2} \mathrm{cos}^2 (\theta) d(-\mathrm{cos}(\theta)) \\
& = -nvp \int_{1}^{0} x^2 dx = nvp \frac{x^3}{3} \bigg|_{0}^{1} \\
& = \frac{1}{3} n p v.
\end{align*}

\noindent
We are almost done, bear with me. Until this point our analysis has assumed that all the particles have the same momentum. To allow a distribution of momentum, we replace $n$ with $n(p)$ (number density of particles with momentum between $p$ and $p + dp$) and integrate over all momentum. Finally we have derived the pressure integral

\begin{equation}
\label{Eq. Pressure}
\boxed{\mathrm{Pressure} = P = \int_{0}^{\infty} \frac{1}{3} n(p) p v dp.}
\end{equation}

\bigskip
\noindent
\textbf{Problem 1:}  For a classical gas the number density  of particles with momentum between $p$ and $p + dp$ is given by the Maxwell distribution
\begin{equation*}
n(p) = \frac{4 n p^2}{\pi^{1/2} (2 m k_B T)^{3/2}} e^{-p^2/(2 m k_B T)},
\end{equation*}
where $m$ is the mass of the gas particle, $k_B$ is the Boltzmann constant, and $T$ is the temperature. Use this distribution and the pressure integral, equation (\ref{Eq. Pressure}), to derive the pressure for a classical gas. Are you surprised?

\bigskip
\noindent
\textbf{Problem 2:}  For an ordinary gas, the pressure is $P=nk_BT$. Hence the pressure vanishes when $T=0$. However, when electrons are compressed to very high densities, many electrons are forced to non-zero momentum even if $T=0$, due to the Pauli exclusion principle, thereby giving rise to degeneracy pressure. The number density  of particles with momentum between $p$ and $p + dp$ for an degenerate electron gas is
\begin{equation*}
n(p) = \frac{2}{h^3}4\pi p^2,
\end{equation*}
where $h$ is plank's constant. The velocity is related to the momentum, in the general relativistic form, as
\begin{equation*}
v=\frac{p}{m\gamma} =\frac{pc^2}{E}=\frac{pc^2}{\sqrt{p^2c^2+m_\mathrm{e}^2c^4}},
\end{equation*}
where $c$ is the speed of light and $m_\mathrm{e}$ is the electron mass. So our pressure integral has the following form
\begin{equation*}
\label{Eq. Degenerate Pressure}
P = \frac{8\pi}{3h^3}\int_0^{p_F}\frac{p^4 c^2}{\sqrt{p^2 c^2+m_\mathrm{e}^2 c^4}}dp.
\end{equation*}
In the integral above, I have truncated the upper limit because the electrons only occupy the lowest momentum states. The upper momentum limit is given by
\begin{equation*}
p_F = \left(\frac{3h^3\rho}{16\pi m_\mathrm{p}}\right)^{1/3},
\end{equation*}
where $m_\mathrm{p}$ is the mass of the proton. Integrate the pressure integral for the relativistic limit, meaning the electrons are moving relativistically such that $pc \gg m_\mathrm{e}c^2$.



\section{Mathematics}
\subsection{Self-Gravitating Barotropic Fluids}
An important situation in astrophysics is static equilibrium, where the pressure gradient balances the gravitational force. This situation can describe a static non-rotating star. In this case the fluid equations have the form
\begin{align*}
\nabla P & = -\rho \nabla \phi\\
\nabla^2 \phi & = 4\pi G\rho.
\end{align*}
We can combine both equations
\begin{align*}
\nabla\cdot \left(\frac{\nabla P}{\rho} \right)= -4\pi G\rho.
\end{align*}
If we further assume the solution is spherically symmetric we have
\begin{align}
\label{Eq. Balance equaution}
\frac{1}{r^2}\frac{d}{dr}\left(\frac{r^2}{\rho}\frac{dP}{dr}\right) = -4\pi G\rho.
\end{align}
We cannot integrate this equation until we have a relation that relates the pressure and the density, $P=P\left(\rho\right)$. Two situations that meet this criteria is isothermal and degenerate gas. Fluids that have an equation of state of the form $P=P\left(\rho\right)$, are called baratropic fluids. Lets write the pressure equation in the following form
\begin{equation}
\label{Eq. Barotropic}
P = K \rho^{(n+1)/n},
\end{equation}
where $K$ and $n$ are constants.

\bigskip
\noindent
\textbf{Problem 3:}
In problem 2 you solved the pressure integral for a degenerate gas, what are $K$ and $n$ for this pressure equation?\\

 It is common practice to scale out unwanted constants, this allows the computation to be simpler without worrying about very large or small constants. Lets introduce new variables $\xi$ and $\theta$,
%\begin{subequations}
%\label{Eq. Scaling}
\begin{align*}
r & = \alpha \xi\\ %\label{Eq. Scaling radius}\\
\rho & = \rho_c \theta^n %\label{Eq. Scaling density},
\end{align*}
%\end{subequations}
where $\alpha$ and $\rho_c$ are constants.

\bigskip
\noindent
\textbf{Problem 4:} Work out the expression for $d/dr$ and $dP/dr$, using equation (\ref{Eq. Barotropic}) and the new variables expressed above. Then substitute your answers into equation (\ref{Eq. Balance equaution}). Your answer should have the following form
\begin{align*}
\frac{1}{\xi^2}\frac{d}{d\xi}\left(\xi^2\frac{d\theta}{d\xi}\right) = \frac{-4\pi G \rho_c^{1-1/n} \alpha^2}{K(n+1)} \theta^n.
\end{align*}
Figure out what value $\alpha$ should have to give the following result
\begin{align}
\label{Eq. Lane Emden}
\boxed{\frac{1}{\xi^2}\frac{d}{d\xi}\left(\xi^2\frac{d\theta}{d\xi}\right) = - \theta^n.}
\end{align}

Lets recap what we have done so far. We used the fluid equations coupled with gravity, assuming spherical symmetry and that the pressure solely depends on the density, to derive a single equation that describes the static configuration of a star, where the pressure force balances the gravitational force. At this point we have a reduced a physical problem into a mathematical problem, which all is left is to solve this second order differential equation. 

\subsection{Initial Conditions and Scaling}
You are almost at the point where you can solve equation (\ref{Eq. Lane Emden}) but before we can turn the numerical crank we need appropriate initial conditions. At the origin of the star we have a central density $\rho_c$ and the force vanishing. This can be formulated as
\begin{align*}
\rho(r=0) &=\rho_c\theta(\xi=0)^n = \rho_c\\
\frac{dP}{dr}(r=0) & \propto \theta\frac{d\theta}{d\xi}(\xi=0) = 0.
\end{align*}
Therefore the appropriate initial conditions for our scaled quantities are
\begin{align}
\label{Eq. Initial Conditions 1}
\theta(\xi=0) &= 1\\
\label{Eq. Initial Conditions 2}
\frac{d\theta}{d\xi}(\xi=0) &=0.
\end{align}.

Once you solve the equation (\ref{Eq. Lane Emden}) you will need to scale the solution to the appropriate units using
\begin{align*}
r & = \alpha \xi \\
\rho & = \rho_c \theta^n.
\end{align*}
Further you can derive the total mass of the star by
\begin{align*}
M &= \int_0^R \rho 4\pi r^2 dr =4\pi\alpha^3\rho_c \int_0^{\xi_1} \theta^n \xi^2 d\xi \\
& = -4\pi\alpha^3\rho_c \int_0^{\xi_1} \frac{d}{d\xi}\left(\xi^2 \frac{d\theta}{d\xi}\right) d\xi\\
& = -4\pi\alpha^3\rho_c \left(\xi^2 \frac{d\theta}{d\xi}\right)\bigg|_0^{\xi_1}\\
M & = -4\pi\alpha^3\rho_c \left(\xi^2 \frac{d\theta}{d\xi}\right)\bigg|_{\xi_1}\\
\end{align*}
where I used equation (\ref{Eq. Lane Emden}) to evaluate the integral and $\xi_1$ corresponds to the radius of the star, $R=\alpha \xi_1$ .

\section{Numerical}
\subsection{Runge Kutta -- First Order Differential Equations}
In class we talked about numerically solving differential equations by using Taylor expansions. Specifically, we mentioned the set of solvers called Runge-Kutta methods. Let's derive the second order Runge-Kutta method. Say we have the following first order differential equation
\begin{equation*}
\frac{dx}{dt} = f(x,t)
\end{equation*}
and we would like to solve it numerically. Starting with the initial value $x\left(t_0\right)$, we want to progress to $x\left(t_0 + h\right)$, where $h$ is our step size. We can perform a Taylor expansion around $t_0+\frac{1}{2}h$ in order to get the value of $x\left(t_0+h\right)$,
\begin{equation*}
x\left(t_0+h\right) = x(t_0+\frac{1}{2}h) + \frac{1}{2}h\frac{dx}{dt}\bigg|_{t_0+\frac{1}{2}h}+\frac{1}{8}h^2\frac{d^2x}{dt^2}\bigg|_{t_0+\frac{1}{2}h} + \mathcal{O}\left(h^3\right).
\end{equation*} 
Similarly we can derive a relation for $x\left(t_0\right)$ by expanding backwards from $x\left(t_0+h\right)$,
\begin{equation*}
x\left(t_0\right) = x(t_0+\frac{1}{2}h) - \frac{1}{2}h\frac{dx}{dt}\bigg|_{t_0+\frac{1}{2}h}+\frac{1}{8}h^2\frac{d^2x}{dt^2}\bigg|_{t_0+\frac{1}{2}h} + \mathcal{O}\left(h^3\right).
\end{equation*} 
Subtracting the above equations and rearranging terms we have the following equation,
\begin{equation*}
x\left(t_0+h\right) = x\left(t\right) + h\frac{dx}{dt}\bigg|_{t_0+\frac{1}{2}h} + \mathcal{O}\left(h^3\right).
\end{equation*}
The term proportional to $h^2$ has dropped out leaving our error to $\mathcal{O}\left(h^3\right)$. However, we are not done yet because we do not know $x\left(t_0+\frac{1}{2}h\right)$ for the expression
\begin{equation*}
\frac{dx}{dt}\bigg|_{t_0+\frac{1}{2}h} = f\left(x(t_0+\frac{1}{2}h), t_0+h\right),
\end{equation*}
we only know the value of $x\left(t_0\right)$. We get around this by using Euler's method, $x(t_0+\frac{1}{2}h) = x(t_0) + \frac{1}{2}hf(x_0, t_0)$ and then substitute in the equation above. It is common practice to label all these terms in the following way, $k_1 = hf(x_0, t_0)$ and $k_2 = hf(x_0+\frac{1}{2}k_1, t_0 + \frac{1}{2}h)$. So that one step in our algorithm takes the form:
\begin{align*}
k_1 & = hf(x, t)\\
k_2 & = hf(x +\frac{1}{2}k_1, t + \frac{1}{2}h)\\
x(t+h) & = x(t) + k_2.
\end{align*}
Thus we have derived the second order Runge-Kutta method. However, we can do even better, although I will not derive it,  by using a fourth order Runge-Kutta method:
\begin{align*}
k_1 & = hf(x, t)\\
k_2 & = hf(x +\frac{1}{2}k_1, t + \frac{1}{2}h)\\
k_3 & = hf(x+\frac{1}{2}k_2, t+\frac{1}{2}h)\\
k_4 & = hf(x+k_3, t+h)\\ 
x(t+h) & = x(t) + \frac{1}{6}(k_1 + 2k_2+2k_3+k_4).
\end{align*}
For your problems in this project use the fourth oder Runge-Kutta algorithm.

\subsection{Runge Kutta -- Second Order Differential Equations}

The Runge-Kutta method we derived in the last section is for first order differential equations, however equation (\ref{Eq. Lane Emden}) is a second order equation. This is not a problem, we can reduce a second order differential equation into a system of first order differential equation. For example, say we want to solve the following differential equation
\begin{equation}
\frac{d^2x}{dt^2} = f(x, \frac{dx}{dt}, t).
\end{equation}
We make the following substitution,
\begin{equation}
y = \frac{dx}{dt},
\end{equation}
so then our second order differential equation splits into a system of two first order differential equations:
\begin{align*}
\frac{dx}{dt} & = y\\
\frac{dy}{dt} & = f(x,y,t).
\end{align*}
We have a system of differential equations, so let's define
\begin{align*}
\mathbf{r} & = (x, y)\\
\mathbf{f} & = (y, f(x,y,t)),
\end{align*}
so that our equations have the following form in vector notation
\begin{equation*}
\frac{d\mathbf{r}}{dt} = \mathbf{f}.
\end{equation*}
The trick is that our new vector $\mathbf{r}$ is only a function of $t$. So we can apply the same derivation from the last section, except our terms are promoted to vector quantities. For example, the first terms in the Taylor expansion of $\mathbf{r}$ is
\begin{equation*}
\mathbf{r}(t+h) = \mathbf{r}(t) + h\frac{d\mathbf{r}}{dt} + \mathcal{O}\left(h^2\right) = \mathbf{r}(t) + h\mathbf{f}(\mathbf{r}, t) + \mathcal{O}\left(h^2\right).
\end{equation*}
So then our fourth order Runge-Kutta scheme for a system of first order differential equations will be
\begin{align}
\label{Eq. Runge-Kutta k1}
\mathbf{k_1} &= \mathbf{f}(\mathbf{r}, t)\\
\label{Eq. Runge-Kutta k2}
\mathbf{k_2} &= \mathbf{f}(\mathbf{r} +\frac{1}{2}\mathbf{k_1}, t+\frac{1}{2}h)\\
\label{Eq. Runge-Kutta k3}
\mathbf{k_3} &= \mathbf{f}(\mathbf{r} +\frac{1}{2}\mathbf{k_2}, t+\frac{1}{2}h)\\
\label{Eq. Runge-Kutta k4}
\mathbf{k_4} &= \mathbf{f}(\mathbf{r} +\mathbf{k_3}, t+h)\\
\label{Eq. Runge-Kutta}
\mathbf{r}(t+h) &= \mathbf{t}+\frac{1}{6}(\mathbf{k_1}+2\mathbf{k_2}+2\mathbf{k_3}+\mathbf{k_4}),
\end{align}
where the $\mathbf{k}$'s are now vector quantities. In the last page of this document there is a partial implementation of this scheme in python, refer to it if you are having trouble coding this scheme.

Your now have all the tools to numerically integrate equation (\ref{Eq. Lane Emden}) with initinial conditions (\ref{Eq. Initial Conditions 1}-\ref{Eq. Initial Conditions 2}), with the fourth order Runke Kutta scheme (\ref{Eq. Runge-Kutta k1}-\ref{Eq. Runge-Kutta}).

\begin{figure}[h!]
\begin{center}
\includegraphics[scale=0.5]{Images/Polytropes}
\end{center}
\caption{Equation (\ref{Eq. Lane Emden}) numerically integrated for $n=0$ and $n=1$. For comparison, the exact solutions are plotted.}
\label{Fig: Polytropes}
\end{figure}

\section{Remaining Problems}
\bigskip
\noindent
\textbf{Problem 5:} Integrate equation (\ref{Eq. Lane Emden}) for $n=0$ and $n=1$. Your integration should start from the center to the radius of the star; the radius is defined where density vanishes. Compare your numerical result with the analytic solution by making a plot, your plot should look like Figure \ref{Fig: Polytropes}. Note: you will notice that the differential equation blows up at the origin. To get around this start the integration one step away from the origin.

\bigskip
\noindent
\textbf{Problem 6:} Integrate equation (\ref{Eq. Lane Emden}) for the degenerate electron gas, remember you showed in problem 3 that pressure equation can be put in terms of $K$ and $n$. Further evaluate the mass of the star, quote your answer in units of stellar mass $M_{\odot}$. It turns out this is the maximum possible mass for a white dwarf, a stellar remnant composed mostly of electrong-degenerate matter with mass comparable to the sun and volume comparable to the Earth. The mass is called the \textit{Chandrasekhar mass limit} and was calculated by the famous astrophysicist Subrahmanyan Chandrasekhar in 1930 when he was 19. Now you have calculated it, very impressive!


\newpage
%\lstinputlisting[language=Python, caption=format, firstline=37, lastline=45,style=FormattedNumber]{Sedov.py}
\lstinputlisting[language=Python, label=RungeKutta, caption=rungekutta.py, style=FormattedNumber]{runge_kutta.py}

\end{document}
